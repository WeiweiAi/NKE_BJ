

% subfigures (a)--(d)

\begin{subfigure}[b]{0.2\textwidth}
\begin{tikzpicture}
% Add an arrow from (0,0) to (3,0) and the above text at the midpoint is $u$ (in red) and below text at the midpoint is $v$( in green)
% 
    \draw[arrow] (0,0) -- (3,0) node[midway, above, red] {$u$} node[midway, below, green] {$v$};
\end{tikzpicture}
\caption{}
\end{subfigure}
\hfill
\begin{subfigure}[b]{0.2\textwidth}
% one junction- a ellipse (0:u_j^i) with five arrows pointing into it
% at the start of each arrow, only showing u_j^0, v_j^1, v_j^2, v_j^3, v_j^4, not the end of the arrow
% not showing the u_j^i at the start of each arrow
% the arrows should point towards the ellipse
% the arrows should be evenly spaced out around the ellipse
\begin{tikzpicture}
  % Arrows first, aimed at node 'zero'
  \node[greencycle, inner sep=3pt] (one) {1:$v_j$};
  \node[red](u0) at ($(one) + (90:2cm)$) {$u_j^0$};
  \node[red](u1) at ($(one) + (162:2cm)$) {$u_j^1$};
  \node[red](u2) at ($(one) + (234:2cm)$) {$u_j^2$};
  \node[red](u3) at ($(one) + (306:2cm)$) {$u_j^3$};
  \node[red](u4) at ($(one) + (18:2cm)$) {$u_j^4$};
    \draw[arrow] (u0) -- (one);
    \draw[arrow] (u1) -- (one);
    \draw[arrow] (u2) -- (one);
    \draw[arrow] (u3) -- (one);
    \draw[arrow] (u4) -- (one);
  % Node last so text is on top
\end{tikzpicture}
\caption{}
\end{subfigure}
\hfill
\begin{subfigure}[b]{0.2\textwidth}
% zero junction- a ellipse (0:u_j^i) with five arrows pointing into it
% at the start of each arrow, only showing v_j^0, v_j^1, v_j^2, v_j^3, v_j^4, not the end of the arrow
% not showing the u_j^i at the start of each arrow
% the arrows should point towards the ellipse
% the arrows should be evenly spaced out around the ellipse
\begin{tikzpicture}
  \node[fill=red!80!black!10, thick, rounded corners, inner sep=3pt] (zero) {0:$u_j$};
  \node[green](v0) at ($(zero) + (90:2cm)$) {$v_j^0$};
  \node[green](v1) at ($(zero) + (162:2cm)$) {$v_j^1$};
  \node[green](v2) at ($(zero) + (234:2cm)$) {$v_j^2$};
  \node[green](v3) at ($(zero) + (306:2cm)$) {$v_j^3$};
  \node[green](v4) at ($(zero) + (18:2cm)$) {$v_j^4$};
    \draw[arrow] (v0) -- (zero);
    \draw[arrow] (v1) -- (zero);
    \draw[arrow] (v2) -- (zero);
    \draw[arrow] (v3) -- (zero);
    \draw[arrow] (v4) -- (zero);
  % Node last so text is on top
\end{tikzpicture}
\caption{}
\end{subfigure}
\hfill
\begin{subfigure}[b]{0.2\textwidth}
\begin{tikzpicture}[node distance=12mm ]
\node[fill=red!80!black!10, thick, rounded corners, inner sep=3pt] (zero) {0:$u_j$};
% a node above the zero node
\node[fill=green!80!black!10, thick, rounded corners, inner sep=3pt, above of=zero] (C) {C:$q_j$};
\draw[arrow] (zero) -- (C);
% two arrows pointing into the zero node from left and another two pointing out of the zero node to the right
\node[green](vL1) at ($(zero) + (180:2cm)$) {$v_j^0$};
\node[green](vL2) at ($(zero) + (150:2cm)$) {$v_j^1$};
\node[green](vR1) at ($(zero) + (0:2cm)$) {$v_j^2$};
\node[green](vR2) at ($(zero) + (30:2cm)$) {$v_j^3$};
\draw[arrow] (vL1) -- (zero);
\draw[arrow] (vL2) -- (zero);
\draw[arrow] (zero) -- (vR1);
\draw[arrow] (zero) -- (vR2);
\end{tikzpicture}
\caption{}
\end{subfigure}
\hfill
\begin{subfigure}[b]{0.2\textwidth}
\begin{tikzpicture}[node distance=12mm ]
% node at the right of the zero and C nodes, in the middle vertically
\node[rectred] (combined) {$q_j$};
% similar layout of the arrows around the combined node
\node[green](vL1b) at ($(combined) + (180:2cm)$) {$v_j^0$};
\node[green](vL2b) at ($(combined) + (150:2 cm)$) {$v_j^1$};
\node[green](vR1b) at ($(combined) + (0:2cm)$) {$v_j^2$};
\node[green](vR2b) at ($(combined) + (30:2cm)$) {$v_j^3$};
\draw[arrow] (vL1b) -- (combined);  
\draw[arrow] (vL2b) -- (combined);
\draw[arrow] (combined) -- (vR1b);
\draw[arrow] (combined) -- (vR2b);
\end{tikzpicture}
\caption{}
\end{subfigure}

