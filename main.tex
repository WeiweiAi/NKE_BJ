% \documentclass[lineno,twocolumn,endfloat,biblatex]{biophys-new}
\documentclass{biophys-new}
\usepackage[utf8]{inputenc}
\usepackage{graphicx}
\usepackage[colorlinks,allcolors=cyan!70!black]{hyperref}
\usepackage{glossaries}
% Uncomment if using biblatex
% \addbibresource{sample.bib}
\newacronym{nka}{NKA}{sodium-potassium ATPase pump}
\title{Model comparison and system analysis of Na+/K+-ATPase using bond graph}
\runningtitle{Biophysical Journal Template} %% For page header

\author[1,*]{First author}
\author[2]{Second author}
\runningauthor{Author1 and Author2} %% For page header

\affil[1]{Institution A, Address A}
\affil[2]{Institution B, Address B}

\corrauthor[*]{abx@xyz.edu}

% \papertype{Letters}
\papertype{Article}
% \papertype{Computational Tools}


\begin{document}

\begin{frontmatter}

\begin{abstract}
Each manuscript must be accompanied by an informative abstract of no more than 300 words. Abstracts should describe the substance of the manuscript in language non-specialists can understand, and must make clear the biological significance of the research. Reference citations are not allowed in the Abstract of a manuscript. 
\end{abstract}

\begin{sigstatement}
Each manuscript must also have a statement of significance or no more than 120 words. Each manuscript must also have a statement of significance or no more than 120 words. Each manuscript must also have a statement of significance or no more than 120 words.
\end{sigstatement}
\end{frontmatter}

\section*{Introduction}

The \gls{nka} uses the hydrolysis of one molecule of ATP to move three sodium ions out of the cell and two potassium ions into the cell,
in both cases against steep concentration gradients. ATP hydrolysis can be thought of as the energy source that tops up the cell's `sodium' battery,
since the the majority of transmembrane transport processes are driven by this sodium gradient.
The NKA pump is therefore critical to cellular function in all cell types.

In this paper we use a bond graph approach to examine a number of models of NKA in which mass, charge and energy are each conserved.
The Post-Albers kinetic scheme {[}.{]} is used for the underlying biochemistry.
We examine a previously published 15-state bond graph
model and then show how a much simpler 6-state model captures the
biophysics of the pump, matches experimental data and, under the
Briggs-Haldane steady state approximation, allows us to derive analytic
expressions that show precisely how the flux depends on the
intracellular and extracellular solutes. We also show how the full bond
graph expression for the steady-state flux relates to a simpler
expression favoured by biochemists

This paper builds on a previous publication {[}.{]} where we developed
models of transmembrane glucose transport using facilitated diffusion
with the GLUT2 transporter and sodium-driven glucose cotransport with
the electrogenic SGLT2 transporter. These models were also developed
using bond graph methods to ensure thermodynamic consistency
(conservation of mass, conservation of charge and conservation of
energy) but for convenience we begin with a summary of the bond graph
approach that underpins all subsequent analysis.

\subsection*{Na+/K+-ATPase physiology and models}


\subsection*{Bond graph basics}

Theoretical manuscripts may include just a \textbf{Methods} section and do not require \textbf{Materials}.


\section*{Methods}


\subsection*{Bond graph models of Na+/K+-ATPase}

\subsubsection{15-state model}

\cite{pan_cardiac_2020}

% include 15state_eqs.tex to get the equations
The equations for conservation of mass are expressed as below.
\begin{align*}
\frac{d q_m^1}{dt}&=v_m^{15}-v_m^{1}      &  \frac{d q_m^2}{dt} &=v_m^{1}-v_m^{2}   &  \frac{d q_m^3}{dt}&=v_m^{2}-v_m^{3}\\
\frac{d q_m^4}{dt}&=v_m^{3}-v_m^{4}       &  \frac{d q_m^5}{dt} &=v_m^{4}-v_m^{5}    &  \frac{d q_m^6}{dt}&=v_m^{5}-v_m^{6}\\
\frac{d q_m^7}{dt}&=v_m^{6}-v_m^{7}       &  \frac{d q_m^8}{dt} &=v_m^{7}-v_m^{8}    &  \frac{d q_m^9}{dt}&=v_m^{8}-v_m^{9} \\
\frac{d q_m^{10}}{dt}&=v_m^{9}-v_m^{10}   &  \frac{d q_m^{11}}{dt}&=v_m^{10}-v_m^{11} &  \frac{d q_m^{12}}{dt}&=v_m^{11}-v_m^{12}\\
\frac{d q_m^{13}}{dt}&=v_m^{12}-v_m^{13}  &  \frac{d q_m^{14}}{dt}&=v_m^{13}-v_m^{14} &  \frac{d q_m^{15}}{dt}&=v_m^{14}-v_m^{15}\\
\end{align*}

The chemical potentials are given by the following.

\begin{align*}
\mu_m^1 & = RT\ln(K_1q_m^1) & \mu_m^2 & = RT\ln(K_2q_m^2) & \mu_m^3 & = RT\ln(K_3q_m^3) & \mu_m^4 & = RT\ln(K_4q_m^4) & \mu_m^5 & = RT\ln(K_5q_m^5) \\
\mu_m^6 & = RT\ln(K_6q_m^6) & \mu_m^7 & = RT\ln(K_7q_m^7) & \mu_m^8 & = RT\ln(K_8q_m^8) & \mu_m^9 & = RT\ln(K_9q_m^9) & \mu_m^{10} & = RT\ln(K_{10}q_m^{10}) \\
\mu_m^{11} & = RT\ln(K_{11}q_m^{11}) & \mu_m^{12} & = RT\ln(K_{12}q_m^{12}) & \mu_m^{13} & = RT\ln(K_{13}q_m^{13}) & \mu_m^{14} & = RT\ln(K_{14}q_m^{14}) & \mu_m^{15} & = RT\ln(K_{15}q_m^{15})
\end{align*}

\begin{align*}
\mu_i^{K+} & = RT\ln(K_i^Kq_i^{K+}) & \mu_i^{Na+} & = RT\ln(K_i^{Na}q_i^{Na+}) & \mu_o^{Na+} & = RT\ln(K_o^{Na}q_o^{Na+}) & \mu_o^{K+} & = RT\ln(K_o^Kq_o^{K+})\\
\mu_i^{ATP} & = RT\ln(K_i^{ATP}q_i^{ATP}) & \mu_i^{ADP} & = RT\ln(K_i^{ADP}q_i^{ADP}) & \mu_i^{P_i} & = RT\ln(K_i^{P_i}q_i^{P_i}) & \mu_i^{H} & = RT\ln(K_i^{H}q_i^{H}) \\
\end{align*}

The flow rates are given by:

\begin{alignat*}{2}
v_m^{1}  &= \kappa_1 \left( K_1 q_m^1 - K_2 q_m^2 K_i^{K} q_i^{K+} \right)
&\qquad
v_m^{2}  &= \kappa_2 \left( K_2 q_m^2 - K_3 q_m^3 K_i^{K} q_i^{K+} \right) \\[6pt]
v_m^{3}  &= \kappa_3 \left( K_3 q_m^3 K_i^{Na} q_i^{Na+} - K_4 q_m^4 \right)
&\qquad
v_m^{4}  &= \kappa_4 \left( K_4 q_m^4 K_i^{Na} q_i^{Na+} - K_5 q_m^5 \right) \\[6pt]
v_m^{5}  &= \kappa_5 \left( K_5 q_m^5 K_i^{Na} q_i^{Na+}
            - K_6 q_m^6 \exp\left( \frac{z_1 F u_m^e}{RT} \right) \right)
&\qquad
v_m^{6}  &= \kappa_6 \left( K_6 q_m^6 - K_7 q_m^7 K_i^{ADP} q_i^{ADP} \right) \\[6pt]
v_m^{7}  &= \kappa_7 \left( K_7 q_m^7 - K_8 q_m^8 \right)
&\qquad
v_m^{8}  &= \kappa_8 \left( K_8 q_m^8 - K_9 q_m^9 K_o^{Na} q_o^{Na+}
             \exp\left( \frac{z_2 F u_m^e}{RT} \right) \right) \\[6pt]
v_m^{9}  &= \kappa_9 \left( K_9 q_m^9 - K_{10} q_m^{10} K_o^{Na} q_o^{Na+} \right)
&\qquad
v_m^{10} &= \kappa_{10} \left( K_{10} q_m^{10} - K_{11} q_m^{11} K_o^{Na} q_o^{Na+} \right) \\[6pt]
v_m^{11} &= \kappa_{11} \left( K_{11} q_m^{11} K_o^{K} q_o^{K+} - K_{12} q_m^{12} \right)
&\qquad
v_m^{12} &= \kappa_{12} \left( K_{12} q_m^{12} K_o^{K} q_o^{K+} - K_{13} q_m^{13} \right) \\[6pt]
v_m^{13} &= \kappa_{13} \left( K_{13} q_m^{13}
             - K_{14} q_m^{14} K_i^{P_i} q_i^{P_i} K_i^{H} q_i^{H} \right)
&\qquad
v_m^{14} &= \kappa_{14} \left( K_{14} q_m^{14} K_i^{ATP} q_i^{ATP}
             - K_{15} q_m^{15} \right) \\[6pt]
v_m^{15} &= \kappa_{15} \left( K_{15} q_m^{15} - K_{16} q_m^{16} \right)
& &
\end{alignat*}

\begin{figure}
\caption{15-state model, adapted from \cite{pan_cardiac_2020}.}
\centering
\includegraphics[width=1\linewidth]{15state.pdf}
\end{figure}

\subsubsection{6-state model and parameter fitting}

% include 6state_eqs.tex to get the equations
\cite{nguyen_structural_2022}
\begin{figure}
\caption{6-state model, based on the proposed scheme in \cite{nguyen_structural_2022}.}
\centering
\includegraphics[width=0.7\linewidth]{6state_2.pdf}
\end{figure}
\input{6state_eqs_2.tex}

\begin{figure}
\caption{6-state model.}
\centering
\includegraphics[width=0.7\linewidth]{6state_1.pdf}
\end{figure}
The equations for conservation of mass are expressed as below.
\begin{align*}
\frac{d q_m^1}{dt}&=v_m^{6}-v_m^{1}      &  \frac{d q_m^2}{dt} &=v_m^{1}-v_m^{2}   &  \frac{d q_m^3}{dt}&=v_m^{2}-v_m^{3}\\
\frac{d q_m^4}{dt}&=v_m^{3}-v_m^{4}       &  \frac{d q_m^5}{dt} &=v_m^{4}-v_m^{5}    &  \frac{d q_m^6}{dt}&=v_m^{5}-v_m^{6}\\
\end{align*}

The chemical potentials are given by the following.

\begin{align*}
\mu_m^1 & = RT\ln(K_1q_m^1) & \mu_m^2 & = RT\ln(K_2q_m^2) & \mu_m^3 & = RT\ln(K_3q_m^3)  \\
\mu_m^4 & = RT\ln(K_4q_m^4) & \mu_m^5 & = RT\ln(K_5q_m^5) & \mu_m^6 & = RT\ln(K_6q_m^6) 
\end{align*}

\begin{align*}
\mu_i^{K+} & = RT\ln(K_i^Kq_i^{K+}) & \mu_i^{Na+} & = RT\ln(K_i^{Na}q_i^{Na+}) & \mu_o^{Na+} & = RT\ln(K_o^{Na}q_o^{Na+}) & \mu_o^{K+} & = RT\ln(K_o^Kq_o^{K+})\\
\mu_i^{ATP} & = RT\ln(K_i^{ATP}q_i^{ATP}) & \mu_i^{ADP} & = RT\ln(K_i^{ADP}q_i^{ADP}) & \mu_i^{P_i} & = RT\ln(K_i^{P_i}q_i^{P_i}) & \mu_i^{H} & = RT\ln(K_i^{H}q_i^{H}) \\
\end{align*}

The flow rates are given by:

\begin{alignat*}{2}
v_m^1 &= \kappa_1\left( K_1 q_m^1 (K_i^{Na} q_i^{Na+})^3 K_i^{ATP} q_i^{ATP}
         - K_2 q_m^2 \right)
&\qquad
v_m^2 &= \kappa_2\left( K_2 q_m^2
         - K_3 q_m^3 K_i^{ADP} q_i^{ADP} \right) \\[6pt]
v_m^3 &= \kappa_3\left( K_3 q_m^3
         - K_4 q_m^4 (K_o^{Na} q_o^{Na+})^3
         \exp\left( \frac{z_2 F u_m^e}{RT} \right)\right)
&\qquad
v_m^4 &= \kappa_4\left( K_4 q_m^4 (K_o^{K} q_o^{K+})^2
         - K_5 q_m^5 \right) \\[6pt]
v_m^5 &= \kappa_5\left( K_5 q_m^5
         - K_6 q_m^6 K_i^{P_i} q_i^{P_i} K_i^{H} q_i^{H} \right)
&\qquad
v_m^6 &= \kappa_6\left( K_6 q_m^6
         - K_7 q_m^7 (K_i^{K} q_i^{K+})^2
         \exp\left( \frac{z_1 F u_m^e}{RT} \right)\right)
\end{alignat*}

The constraint on the thermodynamicparameters of reaction $Rx_m^1$ is as follows:
\begin{equation}
      \label{eq:constraint1}
\dfrac{K_1(K_i^{Na}W_i)^3K_i^{ATP}W_i}{K_2} = \dfrac{1}{(K_{d,Na_i})^2 K_{d,Na_i}^0K_{d,ATP}}
\end{equation}
where $K_{d,Na_i}$ and $K_{d,Na_i}^0$ are voltage-independent dissociation constant and voltage-dependent dissociation constant of intracellular sodium respectively,
$K_{d,ATP}$ is the dissociation constant of intracellular ATP.

The constraint on the parameters of reaction $Rx_m^3$ is as follows:
\begin{equation}
      \label{eq:constraint2}
\dfrac{K_3}{K_4(K_o^{Na}W_o)^3} = \dfrac{(K_{d,Na_o})^2K_{d,Na_o}^0}{1}
\end{equation}
where $K_{d,Na_o}$ are voltage-independent dissociation constant of extracellular sodium.

The constraint on the parameters of reaction $Rx_m^4$ is as follows:
\begin{equation}
      \label{eq:constraint3}
\dfrac{K_4(K_o^{K}W_o)^2}{K_5} = \dfrac{1}{(K_{d,K_o})^2}
\end{equation}
where $K_{d,K_o}$ is the dissociation constant of extracellular potassium.

The constraint on the parameters of reaction $Rx_m^6$ is as follows:
\begin{equation}
      \label{eq:constraint4}
\dfrac{K_6}{K_1(K_i^{K}W_i)^2} = \dfrac{(K_{d,K_i})^2}{1}
\end{equation}
where $K_{d,K_i}$ is the dissociation constant of intracellular potassium.




\subsubsection{Steady-state model and parameter fitting}

\subsubsection{Activity comparison }

\subsection*{Figures and Tables}



\begin{table}[hbt!]
\caption{An example table}
\label{tab:widgets}
\centering

\begin{threeparttable}

\begin{tabular}{c l r}
\hline
Code & Item & Quantity \\\hline
W1 & Widgets\tnote{a} & 42 \\
G35 & Gadgets & 13\tnote{b} \\
\hline
\end{tabular}

\begin{tablenotes}
\item[a] This is a table note.
\item[b] This is another table note.
\end{tablenotes}

\end{threeparttable}

\end{table}

\begin{figure}[hbt!]
\centering
\includegraphics[width=0.6\linewidth]{example-image}
\caption{A figure example.}
\label{fig:view}

\end{figure}

\section*{Results}

\LaTeX{} is great at typesetting mathematics:

Let $X_1, X_2, \ldots, X_n$ be a sequence of independent and identically distributed random variables with $\text{E}[X_i] = \mu$ and $\text{Var}[X_i] = \sigma^2 < \infty$, and let
\begin{equation}
\label{eq:CLT}
S_n = \frac{X_1 + X_2 + \cdots + X_n}{n}
      = \frac{1}{n}\sum_{i}^{n} X_i
\end{equation}
denote their mean. Then as $n$ approaches infinity, the random variables $\sqrt{n}(S_n - \mu)$ converge in distribution to a normal $\mathcal{N}(0, \sigma^2)$. Thus concludes the explanation about Eq.~\ref{eq:CLT}.


You can make lists with automatic numbering \dots

\begin{enumerate}
\item Like this,
\item and like this.
\end{enumerate}

\dots or bullet points \dots

\begin{itemize} 
\item Like this,
\item and like this.
\end{itemize}

\dots or with words and descriptions \dots

\begin{description}
\item[Word] Definition
\item[Concept] Explanation
\item[Idea] Text
\end{description}

An example quotation:

\begin{quote}
Lorem ipsum dolor sit amet, consectetur adipiscing elit, sed do eiusmod tempor incididunt ut labore et dolore magna aliqua. Ut enim ad minim veniam, quis nostrud exercitation ullamco laboris nisi ut aliquip ex ea commodo consequat.
\end{quote}


\section*{Discussion}


\section*{Conclusion}

Sed ut perspiciatis unde omnis iste natus error sit voluptatem accusantium doloremque laudantium, totam rem aperiam, eaque ipsa quae ab illo inventore veritatis et quasi architecto beatae vitae dicta sunt explicabo. 

\section*{Author Contributions}

Author1 designed the research. Author2 carried out all simulations, analyzed the data. Author1 and Author2 wrote the article. 

\section*{Acknowledgments}

We thank G. Harrison, B. Harper, and J. Doe for their help.

% Uncomment if using bibtex (default)
\bibliography{sample}

% Uncomment if using biblatex
% \printbibliography

\section*{Supplementary Material}

An online supplement to this article can be found by visiting BJ Online at \url{http://www.biophysj.org}.

\end{document}
